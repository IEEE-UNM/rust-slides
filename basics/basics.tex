\documentclass{beamer}

% $Header$
\usepackage{listings}
\usepackage{tcolorbox}
\usepackage{graphicx}
\usepackage{xcolor}
\usepackage{../shared/listings-rust}

\definecolor{codegreen}{rgb}{0,0.6,0}
\definecolor{codegray}{rgb}{0.5,0.5,0.5}
\definecolor{codepurple}{rgb}{0.58,0,0.82}
\definecolor{backcolour}{rgb}{0.95,0.95,0.92}

\lstdefinestyle{mystyle}{
    backgroundcolor=\color{backcolour},
    commentstyle=\color{codegreen},
    keywordstyle=\color{magenta},
    numberstyle=\tiny\color{codegray},
    stringstyle=\color{codepurple},
    basicstyle=\ttfamily\footnotesize,
    breakatwhitespace=false,
    breaklines=true,
    captionpos=b,
    keepspaces=true,
    numbers=left,
    numbersep=5pt,
    showspaces=false,
    showstringspaces=false,
    showtabs=false,
    tabsize=2
}

\lstset{style=mystyle, inputpath=src/, language=Rust}

\graphicspath{ {./media/} }

\newtcbox{\inlinecode}{nobeforeafter, tcbox raise base, boxrule=0mm, top=0mm, bottom=0mm, right=0mm, left=0mm}

% This file is a solution template for:

% - Talk at a conference/colloquium.
% - Talk length is about 20min.
% - Style is ornate.



% Copyright 2004 by Till Tantau <tantau@users.sourceforge.net>.
%
% In principle, this file can be redistributed and/or modified under
% the terms of the GNU Public License, version 2.
%
% However, this file is supposed to be a template to be modified
% for your own needs. For this reason, if you use this file as a
% template and not specifically distribute it as part of a another
% package/program, I grant the extra permission to freely copy and
% modify this file as you see fit and even to delete this copyright
% notice.

\mode<presentation>
{
  \usetheme{AnnArbor}
  % \usetheme{CambridgeUS}
  % or ...

  % \setbeamercovered{transparent}
  % or whatever (possibly just delete it)
}


\usepackage[english]{babel}
% or whatever

\usepackage[latin1]{inputenc}
% or whatever

\usepackage[T1]{fontenc}
% Or whatever. Note that the encoding and the font should match. If T1
% does not look nice, try deleting the line with the fontenc.

\title{Common Programming Task}
\subtitle{IEEE42069 Introduction to \ensuremath{\mathrm{Fe_{2}O_{3}}} / \ensuremath{\mathrm{Fe{(OH)}_{3}}}}

\author[]{Tan Hong Kai}
% - Give the names in the same order as the appear in the paper.
% - Use the \inst{?} command only if the authors have different
%   affiliation.

\institute[]{IEEE UNM}
% - Use the \inst command only if there are several affiliations.
% - Keep it simple, no one is interested in your street address.

\date[]{IEEE Workshop}
% - Either use conference name or its abbreviation.
% - Not really informative to the audience, more for people (including
%   yourself) who are reading the slides online

\subject{Programming}
% This is only inserted into the PDF information catalog. Can be left
% out.

% If you have a file called "university-logo-filename.xxx", where xxx
% is a graphic format that can be processed by latex or pdflatex,
% resp., then you can add a logo as follows:

\pgfdeclareimage[height=0.5cm]{university-logo}{../shared/Nottingham}
\logo{\pgfuseimage{university-logo}}


% If you wish to uncover everything in a step-wise fashion, uncomment
% the following command:
% \beamerdefaultoverlayspecification{<+->}

\hypersetup{backref,
     pdfpagemode=FullScreen,
     colorlinks=true}


\title{Rust Basics}

\begin{document}
\begin{frame}
  \titlepage{}
\end{frame}

\section{Variables}
\begin{frame}[fragile]
  \frametitle{Assigning Variable}
  Rust uses the \inlinecode{let} keyword to assign/bind a value to a variable. The general format for assigning a variable is as follows:

\begin{lstlisting}[mathescape=true, language=Rust]
let $\textit{variable\_{}name}$: $\textit{type}$ = $\textit{value}$;
\end{lstlisting}

  Exmaples:
  \lstinputlisting[language=Rust]{variable_assignment1.rs}

  Variables are valid within the scope of a block. A block is anything within braces \inlinecode{\{\}}.
\end{frame}

\begin{frame}[fragile]
  \frametitle{Assigning Variable}
  Often times, the Rust compiler can infer the type of the varible during compile time. Therefore specifying the type is not always nessary.
  \lstinputlisting[language=Rust]{variable_assignment2.rs}

  The compiler can also infer the type when the variable is declared first and then assign the a value at a later time. Just make sure a value is assigned to it before using it.
  \lstinputlisting[language=Rust]{variable_assignment3.rs}
\end{frame}

\begin{frame}
  \frametitle{Mutability and Variables}
  By default, Rust variables are not mutable (cannot change). This is to help programmers to take advantage of Rust's safety and easy concurrency.
  \lstinputlisting{variable_mutability1.rs}

  To make a variable mutable, use the \inlinecode{mut} keyword after the \inlinecode{let} keyword.
  \lstinputlisting{variable_mutability2.rs}
\end{frame}

\begin{frame}
  \frametitle{Variable Shadowing}
\end{frame}

\begin{frame}
  \frametitle{Constants}
\end{frame}

\begin{frame}
  \frametitle{Statics}
\end{frame}

\section{Printing}
\begin{frame}
  \frametitle{println macro}
\end{frame}

\begin{frame}
  \frametitle{Printing Variables}
\end{frame}

\begin{frame}
  \frametitle{Print to Standard Error}
\end{frame}

\section{Primitive Types}
\subsection{Scalar Types}
\begin{frame}
  \frametitle{Integers}
\end{frame}

\begin{frame}
  \frametitle{Floating Point Numbers}
\end{frame}

\begin{frame}
  \frametitle{Boolean}
\end{frame}

\begin{frame}
  \frametitle{Characters}
\end{frame}

\subsection{Compound Types}
\begin{frame}
  \frametitle{Arrays}
\end{frame}

\begin{frame}
  \frametitle{Slices}
\end{frame}

\begin{frame}
  \frametitle{Tuples and Tuple Destruction}
  % Tuple being Hetrogeneous
  % Indexing
  % Destructing
\end{frame}

\section{Functions}
\begin{frame}
  \frametitle{Declaring Functions}
\end{frame}

\begin{frame}
  \frametitle{Function Parameters}
\end{frame}

\begin{frame}
  \frametitle{Function Returns}
\end{frame}

\begin{frame}
  \frametitle{Function Returns}
  % Expressions and Statements (;)
\end{frame}

\section{Control Flow}
\subsection{If Else}
\begin{frame}
  \frametitle{Using Ifs}
\end{frame}

\begin{frame}
  \frametitle{Else If and Else}
\end{frame}

\begin{frame}
  \frametitle{If Else with let}
\end{frame}

\subsection{Loops}
\begin{frame}
  \frametitle{\inlinecode{loop} Keyword}
  % General Syntax
  % Loop Labesl
  % Returning Values
\end{frame}

\begin{frame}
  \frametitle{While Loops}
\end{frame}

\begin{frame}
  \frametitle{For Loops}
  % collections (arrays)
\end{frame}

\section{Comments}
\begin{frame}
  \frametitle{Comments}
\end{frame}

% All of the following is optional and typically not needed.
\appendix
\section<presentation>*{\appendixname}
\subsection<presentation>*{Additional Resources}
\end{document}
