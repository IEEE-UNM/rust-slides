\documentclass{beamer}

% $Header$
\usepackage{listings}
\usepackage{tcolorbox}
\usepackage{graphicx}
\usepackage{xcolor}
\usepackage{../shared/listings-rust}

\definecolor{codegreen}{rgb}{0,0.6,0}
\definecolor{codegray}{rgb}{0.5,0.5,0.5}
\definecolor{codepurple}{rgb}{0.58,0,0.82}
\definecolor{backcolour}{rgb}{0.95,0.95,0.92}

\lstdefinestyle{mystyle}{
    backgroundcolor=\color{backcolour},
    commentstyle=\color{codegreen},
    keywordstyle=\color{magenta},
    numberstyle=\tiny\color{codegray},
    stringstyle=\color{codepurple},
    basicstyle=\ttfamily\footnotesize,
    breakatwhitespace=false,
    breaklines=true,
    captionpos=b,
    keepspaces=true,
    numbers=left,
    numbersep=5pt,
    showspaces=false,
    showstringspaces=false,
    showtabs=false,
    tabsize=2
}

\lstset{style=mystyle}

\graphicspath{ {./media/} }

\newtcbox{\inlinecode}{nobeforeafter, tcbox raise base, boxrule=0mm, top=0mm, bottom=0mm, right=0mm, left=0mm}

% This file is a solution template for:

% - Talk at a conference/colloquium.
% - Talk length is about 20min.
% - Style is ornate.



% Copyright 2004 by Till Tantau <tantau@users.sourceforge.net>.
%
% In principle, this file can be redistributed and/or modified under
% the terms of the GNU Public License, version 2.
%
% However, this file is supposed to be a template to be modified
% for your own needs. For this reason, if you use this file as a
% template and not specifically distribute it as part of a another
% package/program, I grant the extra permission to freely copy and
% modify this file as you see fit and even to delete this copyright
% notice.

\mode<presentation>
{
  \usetheme{AnnArbor}
  % \usetheme{CambridgeUS}
  % or ...

  % \setbeamercovered{transparent}
  % or whatever (possibly just delete it)
}


\usepackage[english]{babel}
% or whatever

\usepackage[latin1]{inputenc}
% or whatever

\usepackage[T1]{fontenc}
% Or whatever. Note that the encoding and the font should match. If T1
% does not look nice, try deleting the line with the fontenc.

\title{Common Programming Task}
\subtitle{IEEE42069 Introduction to \ensuremath{\mathrm{Fe_{2}O_{3}}} / \ensuremath{\mathrm{Fe{(OH)}_{3}}}}

\author[]{Tan Hong Kai}
% - Give the names in the same order as the appear in the paper.
% - Use the \inst{?} command only if the authors have different
%   affiliation.

\institute[]{IEEE UNM}
% - Use the \inst command only if there are several affiliations.
% - Keep it simple, no one is interested in your street address.

\date[]{IEEE Workshop}
% - Either use conference name or its abbreviation.
% - Not really informative to the audience, more for people (including
%   yourself) who are reading the slides online

\subject{Programming}
% This is only inserted into the PDF information catalog. Can be left
% out.

% If you have a file called "university-logo-filename.xxx", where xxx
% is a graphic format that can be processed by latex or pdflatex,
% resp., then you can add a logo as follows:

\pgfdeclareimage[height=0.5cm]{university-logo}{../shared/Nottingham}
\logo{\pgfuseimage{university-logo}}


% If you wish to uncover everything in a step-wise fashion, uncomment
% the following command:
% \beamerdefaultoverlayspecification{<+->}


\title{Embedded Rust}

\begin{document}
\begin{frame}
  \titlepage{}
\end{frame}

\section{Blinky}
\begin{frame}
  \frametitle{Blinky}

\end{frame}

\section{Installing Crates}
\begin{frame}
  \frametitle{Crates, Modules and Packages}
  Rust have features that helps organize code.

  \begin{itemize}
    \item{Packages: A cargo feature that lets you build, test and share crates. A new package is created when \inlinecode{cargo new} is used. It contains a \emph{Cargo.toml} file.}
    \item{Crates: There are two main types, library and binary crates. It can be made out of several modules and there can be multiple crates per package. A binary crates have a main function and compiles into an executable. A library create doesn't have a main function and are meant to share code between projects. Each package can have multiple binary crate but only one library crate.}
    \item{Modules: Modules are used to control privacy of paths and what external crates and modules can see.}
    \item{Paths: It is a way of naming an item, structs, enums, functions and module.}
  \end{itemize}
\end{frame}

\begin{frame}[fragile]
  \frametitle{Installing Crate from crates.io}
  External packages can be installed to the current package. There are a few ways to do this, adding it in Cargo.toml or use \inlinecode{cargo add $package\_name$}.

  There are various crates that are avaiable to \href{crates.io}{crates.io}. A list of amazing packages used for embedded Rust can also be found at \href{https://github.com/rust-embedded/awesome-embedded-rust}{Awesome Embedded Rust}

\begin{lstlisting}[caption={Adding Dependencies in Cargo.toml}, language=toml]
[dependencies]
...
hd44780-driver = "0.4.0"
...
\end{lstlisting}
\end{frame}

\begin{frame}[fragile, allowframebreaks]
  \frametitle{LCD Library}
  Luckily there is a crate for interfacing with the LCD in crates.io and it is called hd44780-driver. To install it type in \inlinecode{cargo add hd44780-driver}.

  To bring it into path do:
\begin{lstlisting}[]
...
use hd44780_driver::HD44780;
...
\end{lstlisting}

  \pagebreak

  Now we can use the library to display ``Hello, world! '' on our LCD:\@

\begin{lstlisting}[]
...
let mut lcd = HD44780::new_4bit(rs, en, d4, d5, d6, d7, &mut delay).unwrap();

// Unshift display and set cursor to 0
lcd.reset(&mut delay).unwrap();

// Clear existing characters
lcd.clear(&mut delay).unwrap();

// Display the following string
lcd.write_str("Hello, world!", &mut delay).unwrap();
...
\end{lstlisting}
\end{frame}

\section{Error Handling}
\begin{frame}
  \frametitle{Types of Errors}
\end{frame}

\begin{frame}
  \frametitle{Irrecoverable Errors}
\end{frame}

\begin{frame}
  \frametitle{Recoverable Errors}

\end{frame}

\section{Enums}
\begin{frame}
  \frametitle{Enums}

\end{frame}

\section{Structs}
\begin{frame}
  \frametitle{Structs}

\end{frame}

\section{RAII and Borrow Checker}
\begin{frame}
  \frametitle{Ownership}

\end{frame}

\begin{frame}
  \frametitle{Borrowing}

\end{frame}

\section{Traits}
\begin{frame}
  \frametitle{Traits}

\end{frame}

\section{Generics}
\begin{frame}
  \frametitle{Generics}

\end{frame}

% All of the following is optional and typically not needed.
\appendix
\section<presentation>*{\appendixname}
\subsection<presentation>*{Additional Resources}
\begin{frame}[allowframebreaks, label={Additional Resources}]
  \href{https://github.com/rust-embedded/awesome-embedded-rust}{Awesome Embedded Rust}
\end{frame}
\end{document}
